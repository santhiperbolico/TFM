\documentclass{article}
\usepackage[utf8]{inputenc}
\usepackage[right=2cm,left=3cm,top=2cm,headsep=0.5cm,footskip=0.5cm]{geometry}

\title{Guión TFM: AGORA}
\author{Santiago Arranz Sanz }
\date{\today}

\usepackage{natbib}
\usepackage{graphicx}
\usepackage{latexsym}
\usepackage{eufrak}
\usepackage{dsfont}
\usepackage{hyperref}
\usepackage{enumerate} 
\usepackage{lscape}
\usepackage{titlesec}
\usepackage{fancyhdr}
\usepackage{color}
\usepackage{cleveref}

\crefformat{section}{\S#2#1#3} % see manual of cleveref, section 8.2.1
\crefformat{subsection}{\S#2#1#3}
\crefformat{subsubsection}{\S#2#1#3}


\newtheorem{teo}{\underline{Teorema}}
\newtheorem{defi}{\underline{Definici\'on}}
\newtheorem{propo}{\underline{Proposici\'on}}
\newtheorem{ejem}{\underline{Ejemplo}}[section]
\newtheorem{prob}{\underline{Problema}}
\newtheorem{lema}{\underline{Lema}}
\newtheorem{obs}{\underline{Observaci\'on}}
\newtheorem{ide}{\underline{Idea}}

\begin{document}

\maketitle


\section{Bibliografía}

\subsection{AGORA}

\cite{kim2016agora, kim2013agora}

\subsection{Historia de ensamblaje de halos de materia oscura}

\subsection{Acreción de satélites}

\section{Análisis de las simulaciones}
\begin{quote}
Empezar los análisis. Lo primero, sacar las propiedades globales de los sistemas galácticos (Scripts \textit{Ms\_Mv.py}), luego la historia de formación estelar, finalmente separar entre estrellas formadas ``in-situ'' vs. ``ex-situ''.
\end{quote}

\subsection{Identificación de los satélistes}
\begin{quote}
 Identificar los satélites más importantes en todos los modelos y analizar sus propiedades en función del tiempo, para cada código.
\end{quote}

\section{Conclusiones}

\newpage
\bibliographystyle{plainnat}
\bibliography{references}

\end{document}
