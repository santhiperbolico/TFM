% Chapter 1 v2

\chapter{Esquema} % Main chapter title

\label{indice} % For referencing the chapter elsewhere, use \ref{Chapter1} 


\section*{Planteamientos a contestar}
\begin{enumerate}
	\item \textbf{Introducción: Luces y sombras del modelo cosmológico estandar.}
	\begin{enumerate}[{1.}1]
		\item Modelo cosmológico estandar, breve descripción y principales logros.
		\item Naturaleza de la matería oscura, teorías y principales ventajas y desventajas de estás
		\item Evolución jerárquica de galaxias en la cosmología estandar.
		\begin{enumerate}[{1.3.}1]
			\item Descripción y teoría.
			\item Simulaciones.
			\item Predicciones, Problemas y Observaciones. Hasta que punto esos problemas se pueden solucionar en el modelo estándar.
		\end{enumerate}
		\item El papel actual de la física barionica como heroe al rescate a los problemas de la cosmología estándar.
		\item Papel de los modelos cosmológicos ante los mismos problemas y observaciones.
		\item ¿En qué punto nos encontramos?¿Por qué no encontramos candidatos a la CDM? Resumen.
	\end{enumerate}

	\item \textbf{Impossible Early Galaxy Problem.}
	\begin{enumerate}[{2.}1]
		\item Descripción del problema de \cite{steinhardt2016impossibly}.
		\begin{enumerate}[{2.1.}1]
			\item Plateamiento de las observaciones analizadas.
			\item Contexto teórico usado en la comparación, ¿es válido?
			\item Relación con otros problemas observados (Impossible Early Quasar Problem, Agujeros negros medianos)
		\end{enumerate}
		\item ¿Cómo abordamos el problema?
		\begin{enumerate}[{2.2.}1]
			\item Los tres posibles frentes: Observaciones mal interpretadas, Papel de la física Bariónica no considerado, Modelo Cosmológico fallido.
			\item Soluciones planteadas en \cite{steinhardt2016impossibly}.
			\item Soluciones de otros autores: Las observaciones están mal interpretadas \cite{behroozi2018mostmassive}, el modelo cosmológico es erroneo (Rh=c)...
			\item Otros trabajos que apoyan y rechazan las hipótesis de \cite{steinhardt2016impossibly}
		\end{enumerate}
	\end{enumerate}
	
	\item \textbf{Observaciones}
	\begin{enumerate}[{3.}1]
		\item Observaciones basadas en el ratio luminosidad - masa halo
		\begin{enumerate}[{3.1.}1]
			\item Descripción de las observaciones de \cite{bouwens2015uv}.
			\item Modelo usado por \cite{steinhardt2016impossibly} para la consversión luz-halo. Descripción del modelo \textit{abundance matching} que enlaza luminosidad con masa estelar. Consideración del enlace entre masa estelar y masa de halo por $M_\odot\sim 70 M\star$.
			\item Medidas de \citep{behroozi2019universemachine} y planteamiento de \cite{behroozi2018mostmassive} en el que plantea el error de \cite{steinhardt2016impossibly}. Análisis de ratio del ratio masa bariónica masa estelar. Rangos de error entre la relación SMHM. Método aplicado a las observaciones de \cite{bouwens2015uv}.
			\item Papel de nuevas observaciones en redhistf altos \citep{wang2019dominant}.
		\end{enumerate}
		\item Observaciones basadas en el \textit{cluster analysis}.
		\begin{enumerate}[{3.2.}1]
			\item Descripción de las observaciones de \cite{hildebrandt2009cars}. Resumen de la metodología.
			\item Posibles errores cometidos en la medida de estas y su incertidumbre.
			\item Dependencia del modelo cosmológico.
			\item Dependencia del volumen de galaxias observadas. ¿Qué pasa si existen muchas más galaxias de las consideradas?
		\end{enumerate}
		\item El papel de nuevas observaciones: \textit{JWST}
	\end{enumerate}
	
	\item \textbf{El papel de la Física Bariónica}
	\begin{enumerate}[{4.}1]
		\item Planteamientos de \cite{steinhardt2016impossibly} y efectos en las medidas.
		\item Una física variante puede implicar una evolución de los ratios de luminosidad, masa estelar y masa de halo.
		\item ¿Cómo considerar estos nuevos ingredientes? El papel de los modelos semi-analíticos
	\end{enumerate}
	
	\item \textbf{El modelo cosmológico}
	\begin{enumerate}[{5.}1]
		\item Descripción de la cosmología usada por \cite{steinhardt2016impossibly}.
		\item Resultados con la variación de parámetros usados.
		\item Simulaciones: Descripción delas Bolshoi y resultados con otras.
		\item ¿Qué pasaría si la teoría considerada no fuera correcta?
		\begin{enumerate}[{5.4.}1]
			\item Consideración de otra naturaleza de la materia oscura: WDM, FDM.
			\begin{enumerate}[{5.4.1.}1]
				\item Descripción
				\item Papel en la evolución galáctica
				\item Simulaciones y resultados
			\end{enumerate}
			\item Variaciones de la teoria: MOND, Rh=c,..
			\begin{enumerate}[{5.5.}1]
				\item Descripción
				\item Papel en la evolución galáctica
				\item Simulaciones y resultados
			\end{enumerate}
		\end{enumerate}
	\end{enumerate}
	
	\item ¿Y ahora qué? Como seguimos a partir de aquí.
\end{enumerate}

\newpage