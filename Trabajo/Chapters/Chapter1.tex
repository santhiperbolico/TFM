% Chapter 1

\chapter{Introducción} % Main chapter title

\label{introduction} % For referencing the chapter elsewhere, use \ref{Chapter1} 

%----------------------------------------------------------------------------------------

% Define some commands to keep the formatting separated from the content 
\newcommand{\keyword}[1]{\textbf{#1}}
\newcommand{\tabhead}[1]{\textbf{#1}}
\newcommand{\code}[1]{\texttt{#1}}
\newcommand{\file}[1]{\texttt{\bfseries#1}}
\newcommand{\option}[1]{\texttt{\itshape#1}}
\newcommand{\lcdm}{$\Lambda$CDM}

%----------------------------------------------------------------------------------------


El origen de este trabajo iba a ser el estudio del modelo de crecimiento de galaxias bajo la influencia por una naturaleza \textit{warm dark matter} (WDM), en contraposición al modelo estandar asociado a la \textit{cold dark matter} (CDM). La finalidad era el estudio de la simulaciones de WDM pero debido al escaso cat\'alogo de estas simulaciones y el estatus de este trabajo se opt\'o por otro enfoque. Dicho enfoque pas\'o desde una perspectiva centrada a las simulaciones num\'ericas a un visi\'on m\'as te\'orica con la esperanza de poder ser traducida en un fut\'uro a una estado m\'as pr\'actica.\\

El objeto de este trabajo es el \textit{Impossible Early Galaxy Problem} (IEGP), definido por primera vez en el paper \cite{steinhardt2016impossibly}, c\'omo la discordancia encontrada en ese mismo paper sobre los datos deducidos de las observaciones de los campos CANDELS, SPLASH y CFHLS sobre la masa de los halos entre los redshift $4<z<7$ y los esperados por el modelo est\'andar (ver \textbf{Figura \ref{fig:stein16_f1}}). En dichas observaciones se encuentran un gran n\'umero de halos m\'uy masivos que quedar\'ian muy por encima de lo esperado por la función de masa de halo deducidad por el modelo cosmol\'ogico \lcdm y el modelo jer\'arquico de crecimiento gal\'actico. 

\begin{figure}[h]
	\begin{center}
	
		\includegraphics[scale=0.5]{Figures/steindhart_fig1}
		\caption{\label{fig:stein16_f1} Figura que representa la funci\'on de masa de halo. En l\'inea continua se muestra la predicci\'on te\'orica sacada de HMFCalc \citep{murray2013hmfcalc} y \cite{sheth2001ellipsoidal} mientr\'as que los marcadores muestran los valores obtenidos a trav\'es de las observaciones estudiadas en \cite{hildebrandt2009cars}, \cite{steinhardt2014uniform}, \cite{bouwens2015reionization} y \cite{bouwens2015uv}.}
		
	\end{center}
\end{figure}


El objetivo final de este trabajo pretende estudiar el problema IEGP desde tres enfoques distintos con la finalidad de clarificar cual puede ser el origen más plausible de las discrepancias observadas entre teoria y observaciones que parecen mostrar la \textbf{Figua \ref{fig:stein16_f1}}. Estos enfoque son:
\begin{description}
	\item[Observacional:] ¿Qué \textit{bias} observacionales pueden estar presentes? ¿Existen otros posibles \textit{bias} observacionales que no se hayan podido tomar en cuenta? ¿Hay otro tipos de sesgos que se hayan podido cometer al calcular la función de masa de halo de las observaciones consideradas? Desde esta perspectiva se pretenderá analizar el problema y determinar si tiene un papel relevante en la discrepancias observadas. También si nuevas observaciones y simulaciones contribuyen de manera positiva o negativa en este punto como por ejemplo los trabajos \cite{wang2019dominant} o \cite{behroozi2019universemachine}.
	
	\item[Física Bariónica:] Son muchos los trabajos que consideran que los procesos físicos de la matería bariónica podrían variar en función del redshift haciendo más eficiente, por ejemplo, los procesos de \textit{feedback} de AGNs que propicián la inhibición del SFR en redhift bajos \citep{finkelstein2015increasing}. Las recetas de la física bariónica juegan un papel fundamental en la relación entre masa estelar \textit{v.s} la masa del halo (SMHMR), veremos si son relevantes en este problema como parece plantear el trabajo de \cite{finkelstein2015increasing} o si por el contrario son suprimibles en el IEGP simplificando un poco el problema.
	
	\item[Modelo Cosmológico:] Por regla general y según ha marcado la experiencia de la historia científica, cuando existe una discrepancia entre observación y teoría suele ser el segundo factor el que está equivocado. Son muchos los logros del \lcdm pero también existen muchos problemas no explicados por este modelo y el IEGP parece convertirse en otro más. Daremos un repaso de los problemas que parece no explicar el \lcdm e intentaremos estudiar el IEGP considerando otros modelos cosmógicos y naturaleza de materia oscura que se encuentran sobre la mesa de teorías aceptadas o con el estatus de encontrase en consideración debido a las posibles carencias del \lcdm.
\end{description}


En esta primera sección daremos un pequeño resumen de los principales puntos del paper de \cite{steinhardt2016impossibly} comparando de manera conjunta los resultados con los que se derivarían de las nuevas medidas ofrecidas por \cite{behroozi2019universemachine}, en un intento de ver si datos más actualizados pueden contribuir a solucionar o agravar las discrepancias observadas. En las siguientes secciones abordaremos de manera independiente los tres enfoques explicados terminando con una última sección dedicada a las conclusiones y a los posibles fututos trabajos que se podrían realizar en el campo de las simulaciones en aras de clarificar este problema . Pero lo primero es explicar el problema y allá vamos.


\section{The Impossible Early Galaxy Problem}

Existe un consenso bastante amplio en el que en un escenario basado en el modelo cosmológico \lcdm las grandes masas finales de los halos predichos por la función masa de halo cambian rápidamente entre los redshift 8 y 4, cuyos halos que contienen a las galaxias más masivas virializan hacia z=4 \citep{steinhardt2016impossibly}. Sin embargo, no se han encontrado evidencias de esa evolución esperada entre los rangos de redshift mencionados, por ejemplo, trabajos como el \cite{finkelstein2015increasing} y \cite{finkelstein2015evolution} han deducido una indepencia del valor de la magnitud UV característica $M^*_{UV}$ de la función de  paraetrización de Scheter con respecto al redshift en los rangos z=7-4, por tanto, sin encontrar signos de una evolución esperada entre las galaxias de estos redshift. Dichos trabajos encontraron un mayor número de galaxias brillantes en UV de lo esperado en redshift z=7 contrario a lo esperado y que concuerda con el resultado estudiado aquí del paper de \cite{steinhardt2016impossibly} y que se puede visualizar en la \textbf{Figura \ref{fig:stein16_f1}}. En \cite{steinhardt2016impossibly} compara las funciones de masa de halo obtenidas de los trabajos de \cite{hildebrandt2009cars}, \cite{steinhardt2014uniform}, \cite{bouwens2015reionization} y \cite{bouwens2015uv} calculadas basándose en las observaciones de los estudios de CANDELS\footnote{\cite{grogin2011candels}}, CFHTLS\footnote{\cite{hildebrandt2009cars}} y SPLASH\footnote{\cite{capak2012splash}} y usando tres metodologías distintas que explicaremos en la siguiente subsección, con las funciones de masa de halo predichas por modelos teóricos obtenidos de la herramienta \textit{HMFCalc} \citep{murray2013hmfcalc} usando como base el trabajo de \cite{sheth2001ellipsoidal} y que explicaremos con un mayor detalle en otra subsección más adelante. Las discrepancias que se pueden observar en la \textbf{Figura \ref{fig:stein16_f1}} son muy notables y son estudiadas en \cite{steinhardt2016impossibly}. Aquí haremos un pequeño resumen de estas explicaciones aplicando las nuevas métricas obtenidas de \cite{behroozi2019universemachine} pero antes para poder comprender mejor los posibles \textit{bias} observacionales considerados u omitidos explicaremos en mayor detalle como se han obtenido dichas medidas de la función de masa de halo de las observaciones.

\subsection{Datos Observacionales}


\section{Tareas}
\begin{enumerate}
	\item Repasar \cite{steinhardt2016impossibly}. Revisar resumen creado junto con \cite{arranz2015finkelstein} para sacar principales puntos y plasmarlos. 
	\item Redactar resumen de la cuestión principal y posibles subcuestiones.
	\item Repasar las posibles explicaciones dadas por \cite{steinhardt2016impossibly}
	\item Leer \cite{behroozi2019universemachine} donde hace un análisis del problema (por lo menos del de \cite{finkelstein2015increasing}).
	\item Representar los calculos realizados en \citep{behroozi2019universemachine}.
\end{enumerate}

