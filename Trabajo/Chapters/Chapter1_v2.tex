% Chapter 1 v2

\chapter{Luces y sombras del modelo cosmológico estándar} % Main chapter title

\label{introduction} % For referencing the chapter elsewhere, use \ref{Chapter1} 

%----------------------------------------------------------------------------------------

% Define some commands to keep the formatting separated from the content 
\newcommand{\keyword}[1]{\textbf{#1}}
\newcommand{\tabhead}[1]{\textbf{#1}}
\newcommand{\code}[1]{\texttt{#1}}
\newcommand{\file}[1]{\texttt{\bfseries#1}}
\newcommand{\option}[1]{\texttt{\itshape#1}}
\newcommand{\lcdm}{$\Lambda$CDM }

%----------------------------------------------------------------------------------------

\section{Modelo cosmológico estándar, breve descripción y principales logros.}

\section{Naturaleza de la materia oscura, teorías y principales ventajas y desventajas de estás}

\section{Evolución jerárquica de galaxias en la cosmología estándar.}

\subsection{Descripción y teoría.}
\subsection{Simulaciones.}
\subsection{Predicciones, Problemas y Observaciones. Hasta que punto esos problemas se pueden solucionar en el modelo estándar.}

\section{ El papel actual de la física barionica como héroe al rescate a los problemas de la cosmología estándar.}

\section{Papel de los modelos cosmológicos ante los mismos problemas y observaciones.}

\section{¿En qué punto nos encontramos? ¿Por qué no encontramos candidatos a la CDM? Resumen.}

