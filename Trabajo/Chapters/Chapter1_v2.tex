% Chapter 1 v2

\chapter{Luces y sombras del modelo cosmológico estándar} % Main chapter title

\label{introduction} % For referencing the chapter elsewhere, use \ref{Chapter1} 

%----------------------------------------------------------------------------------------

% Define some commands to keep the formatting separated from the content 
\newcommand{\keyword}[1]{\textbf{#1}}
\newcommand{\tabhead}[1]{\textbf{#1}}
\newcommand{\code}[1]{\texttt{#1}}
\newcommand{\file}[1]{\texttt{\bfseries#1}}
\newcommand{\option}[1]{\texttt{\itshape#1}}
\newcommand{\lcdm}{$\Lambda$CDM }
\newcommand{\addcite}{\textcolor{red}{[CITA]}}

%----------------------------------------------------------------------------------------

\section{Modelo cosmológico estándar, breve descripción y principales logros.}
Dentro del modelo cosmológico estándar \lcdm, el cual queda restringido al marco de la relatividad general de Einstein con una componente de energía oscura correspondiente a un fluido con parámetro de estado de $\omega=-1$, otra bariónica  y a una 
naturaleza fría de la materia oscura denominada \textit{Cold Dark Matter}, existe un gran consenso en la teoría de la formación y evolución de estructuras del Universo \addcite, siendo su origen las fluctuaciones cuánticas que dejó a descubierto el rápido proceso de inflacción ocurrido pocos \textcolor{red}{nano segundos} después del Big Bang y que crecieron en una primera fase de manera lineal producto de procesos de autoalimentación de las fluctuaciones existente \addcite, sin embargo existe un límite donde esa descripción es validad y entran en una fase de crecimiento no lineal. Es en este punto donde el avance de las últimas décadas ha arrojado más luz, sirviéndose de simulaciones de N-cuerpos cada vez más depurados (Millenium, Bolshoi,...)\addcite que nos permite ver como se acumula la materia oscura y modelos semi-analíticos (de Lucia,...)\addcite donde se añaden las recetas de la física bariónica, nos permiten simular escenarios hasta el universo local pudiendo ver desde las estadísticas de grandes números \addcite hasta las características de rotación de las galaxias espirales en $z\sim 0$ \addcite.\\

En el contexto de este paradigma de modelo estándar más un modelo jerárquico de evolución de galaxias, donde se considera que la evolución de las éstas parte de pequeñas galaxias que van sufriendo diversas fusiones con otras galaxias a lo largo de su vida hasta llegar formar grandes galaxias, agrupándose en grandes cúmulos \addcite .Algunas de las predicciones o grandes logros del modelo cosmológico pueden ser la detección del fondo cósmico de microondas a una temperatura de $3^\circ\mathrm{K}$ \addcite, la abundacia de elementos ligeros como el deuterio $D$, el helio tres $^3He$, la fracción de helio cuatro $^4He/H\sim 0.25$ y del litio siete $^7Li$ \addcite , la precisión de los parámetros cosmológicos deducidos por la teoría y observados como la fracción bariónica $\Omega_b$, la abundacnia de neutrinos y fotones, la precisión en el parámetro de expansión del universo y aceleración, etc \addcite.\\

Procesos como la observación de una gran abundancia de galaxias del tipo \textit{``temprano''} \footnote{El termino temprano está traducido del término ``\textit{earl-type}'' en ingles en donde hace referencia a las galaxias principalmente del tipo elíptico, ya que en época de Hubble se consideraban las galaxias evolucionabas de elipticas a espirales las cuales se denotaban como del tipo \textit{``tardío''} o en inglés \textit{``old-type''}, cosa que se demostró no ser cierta siendo precisamente una prueba que en épocas de $z\sim 2$ la abundancia de galaxias del tipo elíptico eran mucho menos abundantes que en la actualidad.} en el Universo local mientras que en épocas $z >2$ su abundancia es mucho menos notoria observándose una mayor cantidad de galaxias del tipo \textit{``tardío''} \addcite respaldan el modelo jerárquicos de evolución de galaxias \addcite. Otras de éstas observaciones es la abundancia de galaxias satélites (con respecto a otros modelos de materia oscura),......



\section{Naturaleza de la materia oscura, teorías y principales ventajas y desventajas de estás}

\section{Evolución jerárquica de galaxias en la cosmología estándar.}

\subsection{Descripción y teoría.}
\subsection{Simulaciones.}
\subsection{Predicciones, Problemas y Observaciones. Hasta que punto esos problemas se pueden solucionar en el modelo estándar.}

\section{ El papel actual de la física barionica como héroe al rescate a los problemas de la cosmología estándar.}

\section{Papel de los modelos cosmológicos ante los mismos problemas y observaciones.}

\section{¿En qué punto nos encontramos? ¿Por qué no encontramos candidatos a la CDM? Resumen.}

