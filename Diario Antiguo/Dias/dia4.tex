
\section*{Resumen 13/03/2019}
Las conclusiones sacadas de la reunión del miércoles fue concretar más sobre el tema, pueden existir bias observacionales en el problema planteado en \cite{steinhardt2016impossibly} y hay que tenerlos en cuenta. Para ver dichos problemas de la página de \url{http://cdsads.u-strasbg.fr/} me he descargado las citas del árticulo para ver las críticas que ha suscitado. Una vez visto esto, el IEGP pudiera estar relacionado con otro tipo de problemas que si tienen una mayor historia, como:
\begin{description}
\item[SFR en altos redshift] Se haobservado que en altos redshift el SFR tiene mayores valores que en el universo local, lo que por teoría de formación de galaxia no deería de darse. ¿Qué motivos hay? puede ser consecuencia de un bias observacional al estar solo observando las galaxias más activas y luminosas o por efecto de los AGNs, cuando empiezan su actividad frenan la formación estelar. Para concretar más en el tema repasar el artículo de Pérez 2008 (añadir cita)
\item[Existencia primero de las galaxias masivas] Se ha observado que las galaxias más masivas se presentan en altos redshift y las menos masivas a menor redshift, esto parece sugerir que son las más masivas las que se crean antes y la menos masivas después. Esto sería un cambio en el modelo de formación galáctica jerárquica por colapso elipsoidal. Como idea loca puede ser que se formen galaxias mas masivas y en sus fusiones entre ellas se formen por fragmentación las menos masivas y que estas se vuelvan a fusionar entre sí para formar las galaxias mas masivas, entrando en ciclo de fusiones que ha de acaar con solo galaxias masivas. Ver artículo de Bower (añadir cita)
\item[Agujeros Negros Medianos] Se han presenciado agujeros negros pequeños y masivos, pero de esclas intermedias no se han visto, ¿por qué? Pueda ser una pata del problema de que se vean agujeros negros supermasivos a altos redshift.
\end{description}

Estos tres problemas, junto con el IEGP, pueden formar un problema de único que estudiar. La idea es estructurar el trabajo en tres partes diferenciadas.
\begin{enumerate}
\item La primera pata sería exponer los argumentos en contra a dichos problemas, es decir, posibles bias observacionales que no se estén teniendo en cuenta y que dicho problema pueda no existir.
\item La segunda pata sería, descarando los bias observacionales, posibles fallos en las observaciones. En el caso del IEGP la estimación del SHMR por la luminosidad pueda no ser cierta porque se está tomando como constante en el tiempo. EN el artículo de Rod-Puebla (añadir cita) cita estudia dicho problema y quizás se pueda estimar con los datos de CANDELS y el método de Mireia Montes - Nacho Trujillo (añadir cita) de la IL para estimar la masa del halo.
\item Considerando que las observaciones son correctas, quizás el problema sea de la teoría. Considerar las diferentes soluciones: WDM, otros modelos cosmológicos, etc...
\end{enumerate}

Por último hablamos de la formación estelar anterior a la formación de los halos como posible solución. Esta formación estelar de POP III podría evolucionar tan rápido que diese formación estelar de POP II. Hemos quedad en estudiar dicho problema ya que no hay muchos trabajos sobre ello y se está empezando a hablar sobre ello.


\section*{Día 03/04/2019}
