
\section*{14/06/2019}

Después de leer varioas artículos en los que trata dicho problema y derivados he llegado a la conclusión que el problema base es la discordancia del modelo actual de fusión jerárquica con la falta de observación de galaxias transitorias entre los halos iniciales y las galaxias más masivas en los redshift $z\sim 4-6$. Veamos algunos resumenes de los artículos principales.

\subsection*{The Impossible Early Galaxy Problem}
\citep{steinhardt2016impossibly} Según el paradigma actual de la fusión jerárquica de galaxias en el modelo cosmológico estandar, en torno a los redshift $z\sim 4-6$ ha de existir la transición de las galaxias más masivas desde los halos iniciales que acretan masa a los ultimas fases de evolución bariónica vista en las galaxias con formación estelar y \textit{quasars}. Sin embargo, ninguna evidencia ha sido encontrada en muchos estudios a alto redshift como el CFHTLS, CANDELS y el SPLASH, los primeros estudios para probar la masa alta final en estos redshift. Considerando que el ratio masa halo- masa estelar (SMHMR) estimado a bajo redshift permaneciera constante en $z\sim 6-8$, CANDELS y SPLASH darían mayores ordenes de magnitud de número de halos de masa $M\sim 10^{12-13}M_\odot$ que los que se podrían haber formado a esos redshift, esto se conoce como el problema de la galaxia masiva temprana imposible. En el artículo de \cite{steinhardt2016impossibly} consideran los posibles errores sistemáticos que puedan explicar esta contradicción de teoría y observaciones en los modelos de síntesis estelar usados para estimar los parámetros físicos  y en los escenarios posibles de formación galáctica. Es posible que las incertidumbres desconocidad reduzcan la disparidad entre observaciones y simulaciones de CMD tomando una visión conservadora de las observaciones,aun así existirían tensiones considerables con la teoría.\\


Existe un consenso bastante amplio en la distribución de masa y redshift de los halos producidos en el colapso inicial de las pequeñas fluctuaciones de densidad en el universo temprano y el modelo de fusión jerárquico. Para la cosmología estandar la función de masa es sencilla decalcular. Este consenso se basa en la idea de la rápida evolución en la densidad de halos masivos en $z>4$ que deberían ser observacionalmente evidente en las funciones de masa y luminosidad galácticas. Hasta hace poco el catálogo de galaxias a redshift $z>6$ era limitádo y sesgado a las galaxias más brillantes, galaxas individuales masivas y \textit{quasars}. Sin embargo con los nuevos estudios como el CANDELS o el SPLASH se ha podido probar la función de luminosidad y de masa para galaxias en el rango de $z\sim 4-8$, lo que nos permite calcular la función de masa del halo correspondiente a dicho rango.\\

Trabajos como el de \cite{finkelstein2015increasing} muestra las tensiones ocasionadas en $z>4$ entre la evolución esperada de la función de masa del halo y las funciones de luminosidad y de masa de las galaxias. \\

\subsubsection*{An increasing stellar baryon fraction in bright galaxies at high redshift.}
En el artículo de \cite{finkelstein2015increasing} se centran en el estudio de la función de luminosidad en el UV utilizando observaciones realizadas por el HST con el detector WFC3 y el instrumento IRAC del Spitzer en una muestra de 7500 galaxias con una magnitud alrededor de $M_{UV}^*\simeq -21$ sobre el rango de redshift $z\sim 4-8$, ya que pretenden entender los procesos físicos que regulan la la abundancia de galaxias brillantes en el universo distante. \\

Gracias a los últimos estudios de las pasadas décadas se an facilitado las medidas de la función de luminosidad en el UV, la cual cuantifica las abundacias relativas de galaxias sobre un amplio rango dinámico en luminosidad. Cono laluz UV es un indicador de la actividad de formación estelar, la intefral de la función de luminosidad UV es un indicador de la densidad de formación estelar. La función de luminosidad es parametrizada con la función de Schechter de la forma que se ajusta a una ley de potencias en luminosidades bajas y cae exponencialmente en luminosidades altas. Dichos parámetros se corresponden con la luminosidad característica $M_{UV}^*$, la pendiente $\alpha$ y la normalización $\phi^*$. Estudios anteriores vieron que estos parámetros evolucionaban según el redshift, decayendo tanto $M_{UV}^*$ como $\alpha$ según incrementa el redshift. Esta ``evolucion luminosa'' de la función de luminosidad fue ampliamente aceptada como ajuste general de la tendencia observable en la evolución de la densidad de la formación estelar. Sin embargo, trabajos más recientes mostraron que laimagen dibujada era incompleta. Las primeras evidencias vienen de trabajos donde muestran un mayor número de lo esperado de galaxias brillantes sobre redshift $z=7$. \\

Estudios posteriores confirmaron este exceso de galaxias brillantes y concluyeron que no era un efecto atribuible a lentes gravitacionales. Se calculo la función de luminosidad UV y se observo que era contraria a los resultados derivados de conjuntos de datos más pequeños, la luminosidad característica $M_{UV}^*$ era significativamente independiente del redshift entre 4,5,6 y 7. La continuidad del valor de $M_{UV}^*$ se rompe al llegar a $z\sim 8$, donde este cae. Además, mientras que las galaxias en general llegan a ser menos comunes en redshift altos - consistente con la caída en la densidad de la formación estelar- las galaxias brillantes permanecieron relativamente comunes en el universo distante. \textcolor{red}{Esto puedeser posible por un bias observacional, sin embargo no creo que el bias observacional puede explicar que el número de galaxias brillantespermanezca constante, aunque la evolución de la densidad en numero de galaxias en el universo distante no lo ha tratado aún.}\\

En el artículo buscarán los límites de las propiedades físicas en las galaxias brillanes UV distantes e intentarán comprender como ellas mantienen altos niveles de de formación estelar. Los parámetros cosmológicos usados en \cite{finkelstein2015increasing} son los dados por WMAP7, donde $H_0=70.2$ km s$^{-1}$ Mpc $^{-1}$, $\Omega_m=0.275$ y $\Omega_\Lambda=0.725$. De las 7500 galaxias de muestra se centran en 150,75,28,18 y 3 galaxias de $M_{UV}<-21$ en redshift $z=4,5,6,7$ y 8 respectivamente, donde los intervalos de redshift son de un grosor de $\Delta z=0.5$ centrados en los redshift marcados.\\

Tras una criba de galaxias tras los ajustes de la distribución espectral de energías (SED) finalmente el estudio se quedan solo con galaxias entre los redshift $z\sim 4-7$. Los parámetros de la mediana del ajuste con un intervalo de confianza de $1\sigma$ no evolucionan significativamente, es decir, podemos hablar en general en redshift $z\sim 4-7$ que las galaxias son moderadamente masivas $\log[M_\star/M_\odot]\sim 9.6-9.9$, algo jóvenes con edades $<100$ Myr, tienen atenuación debida al polvo nada despreciables con $E(B-V)=0.07-0.13$ y alto ratio de formación estelar SFR$\sim 40-60\ M_\odot$ yr$^{-1}$. \textcolor{red}{Lo mas destacable es que no parece haber una relación entre redshift y el valor de estos parámetros, quizás un ligero aumento del SFR, Masa estelar y de la atenuación debida al polvo según decrece el redshift, aunque en muchos casos entran dentro del intervalo de confianza Tabla \ref{tab:finkelstein1}}.\\

\begin{table}[h]
\begin{center}
\begin{tabular}{lrrrrr}
\hline \hline\\
Redshift & Number & $\log(M_\star/M_\odot)$ & Age & $E(B-V)$ & SFR\\
	&	&	&	(Myr)	&	& $(M_\odot)$ yr$^{-1}$\\
\hline\\
z=4 & 94 & 9.86$\pm$0.04 & 44$\pm$2 & 0.13$\pm$0.01 & 56$\pm$4\\
z=5 & 46 & 9.80$\pm$0.06 & 35$\pm$2 & 0.12$\pm$0.02 & 52$\pm$10\\
z=6 & 19 & 9.78$\pm$0.07 & 40$\pm$4 & 0.07$\pm$0.02 & 40$\pm$8\\
z=7 & 14 & 9.64$\pm$0.13 & 29$\pm$8 & 0.09$\pm$0.02 & 41$\pm$9\\
\hline
\end{tabular}
\caption{\label{tab:finkelstein1} Medianas de las propiedades fisicas de galaxias con $M_{UV}<-21$ sacada de \cite{finkelstein2015increasing}}
\end{center}
\end{table}

El paper \cite{finkelstein2015increasing} confirma que las cantidades de polvo observadas en las galaxias masivas/ brillantes en $z=4-7$ son equivalentes. Además, las galaxias menos brillantes y masivas parecen tener menos cantidad de polvo a altos redshift, lo que no es cierto en galaxias brillantes, que confirmó ALMA detectando emisión de polvo en redshift $z\sim 5-7.5$. Si existiera una evolución de la atenuación debida al polvo en las galaxias mas brillantes según evolucione el redshift  podría haber llevado a seleccionar una masa estelar inferior dada una magnitud UV, pero esto no ocurre lo que nos lleva al resultado de que la masa estelar de dichas galaxias brillantes permanecen constante sobre dicho intervalo de redshift. \\

Una pregunta interesante ess como las galaxias brillantes en UV en redshift 2-3 están relacionadas con las galaxias brillantes en redshit 6-8. Es decir, desde el punto de vista de la fusión jerárquica, ¿las galaxias brillantes de redshift bajos son descendientes de las galaxias brillantes de redshift altos? ¿Son las galaxias de resdshift altos progenitores de las de redshift bajos? Además, la fusión de galaxias complica la comparación directa basada en el número d edensidad de galaxias en cada redshift. Algunos estudios han intentado comparar galaxias en diferentes redshift con el mismo número de denisidad mostrando que co


Las galaxias $M_{UV}<-21$ en $z\sim 2-3$ son los descendientes plausibles de galaxias de alto redshift más débiles de magnitud.


