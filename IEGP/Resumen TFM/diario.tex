\section*{Planteamientos iniciales de cada día de trabajo}
Notas que pretenden estructurar el trabajo día a día
\subsection*{Agosto 2019}
A falta de terminar de analizar el paper de \cite{finkelstein2015increasing} y de hacer una lista de las dudas surgidas y propuestas a dar ya he leido el artículo de \cite{wang2019dominant} sobre las galaxias a redshift $z>3$ descubiertas por ALMA las cuales parecen ser abundantes y masivas y no detectables por el HST. Esto tumbaría la susposición de \cite{finkelstein2015increasing} de la no existencia o poca abundancia de las galaxías sub-milimétricas lo que modificaría la función de luminosidad y por tanto la relación de masa halo - masa estelar ya que en el paper de \cite{finkelstein2015increasing} se basa principalmente en ella. La propuesta de lo que quiero hacer hoy es lo siguiente:
\begin{enumerate}
\item Terminar el análisis de \cite{finkelstein2015increasing}. $\surd$
\item Como encaja las nuevas observaciones de \cite{wang2019dominant} en el paper de \cite{finkelstein2015increasing} $\surd$
\item Plantear como incluir los planteamientos de \cite{wang2019dominant}: 
\begin{enumerate}[i.]
\item Replantear la función de luminosidad de \cite{finkelstein2015increasing}.
\item Como encaja en el modelo jerárquico \citep{bower2006breaking}.
\item Las simulaciones de RAMSES nos pueden dar un orden distinto a la masa de los halos que encajen de una manera distinta de con la función de luminosidad. 
\end{enumerate}
\end{enumerate}

\subsection*{Septiembre 2019}
\subsubsection*{Semana 2: 9-15}
El artículo de \cite{wang2019dominant} parece indicar que algunas de las suposiciones en las que se basan los métodos de cálculo de las masa de halo están equivocados, por lo que el escenario más factible para la resolución del problema de \cite{steinhardt2016impossibly} sea el error observacional. Las tareas para esta semana son las siguientes:
\begin{enumerate}
\item Terminar el análisis de \cite{steinhardt2016impossibly}. $\surd$
\item Leer el artículo de \cite{sheth2001ellipsoidal} y \cite{murray2013hmfcalc} que trata las estimaciones teóricas usadas para el cálculo de la función de masa de halo. Empezar con el resumen de \cite{sheth2001ellipsoidal}.
\item Terminar el punto 3 de agosto.
\item Escribir a Santi con los avances el domingo día 15 y plantear con él una reunión.
\end{enumerate}

\subsubsection*{Reunión lunes 30}
En la reunión con Santi vamos a retomar el proyecto del TFM. Durante el verano Santi me comento que los modelos MOND podrían resolver el problema de la función de masa de halo cargándose el concepto de materia oscura. La idea supongo que es encajar este nuevo modelo cosmológico en el problema, pero ¿cómo?. ¿Existén predicciones de lamasa de galaxias según el redshift para este modelo? ¿Simulacones de formación galáctica?.
\begin{enumerate}
\item Las observaciones de \cite{wang2019dominant} parecen incrementar el problema. La nuevas observaciones de casos extremos de SMG en el campo CANDELS parecen suponer que los halos de tendrían masas superiores empeorando el problema hasta ahora.
\item El paper de \cite{behroozi2019universemachine} parece dar ajustes en un rango de $0<z<10$ de los SFR, Ratios de UV Masa estelar, funcion de UV,etc... a través de simulacione de MCMC con 400k de horas de CPU. Estos parámetros podrían aplicarse para el cálculo teórico y el ajuste observacional de la \textbf{Figura \ref{fig:steindhart_fig1}} de \cite{steinhardt2016impossibly}. Puede ser un buen comienzo del TFM para introducir el problema. Hay que recalcar que \cite{steinhardt2016impossibly} considera la varición de algunos de estos parámetros descartando sus efectos por insuficientes, pero quizás merece otra reconsideración como introducción.
\item Consideración del MOND. Esta teoría no es muy aceptada, preguntarle porque tiene que ser considerada. Existen ejemplos de galaxias sin DM, puede ser una prueba de que MOND puede funcionar \citep{van2018galaxy}. Por otro lado habría que resultados hay disponibles para el encaje con nuestro problema, ¿existen simulaciones disponibles? ¿funciones d emasa estelar que encajen con las observacones? Los pocos estudios que he visto de Kroupa hablan de galaxias enanas.
\item Por último las simulaciones tratadas por \cite{behroozi2019universemachine} y por \cite{finkelstein2015increasing} son las de Bolshoi, quizás sea interesante poder sustituirlas por la de RAMSES. Los datos tratados de simulaciones por \cite{steinhardt2016impossibly} son los de Millenium, con ratios de masa halo - masa estelar muy pequeños (\textcolor{red}{esto podría tratarse como prueba de que no es necesario la materia oscura?????}) y los de Illustris con pocos halos de masa estelar superior a $10^{10.5}M_\odot$. Las simulaciones de RAMSES se pueden utilizar en el contexto de MOND ????
\end{enumerate}

El problema de \cite{steinhardt2016impossibly} es un problema global de la cosmología. Tiene relación con SBF de \cite{finkelstein2015increasing}, el problema de downsizing, la formación de grandes ahujeros negros muy temprano e incluso con la no detección de agujeros negros intermedios. Parece indicar un problema en el modelo que ha de ser tratado.