\section*{Planteamientos iniciales de cada día de trabajo}
Notas que pretenden estructurar el trabajo día a día
\subsection*{Agosto 2019}
\subsubsection*{Día 15 - 18}
A falta de terminar de analizar el paper de \cite{finkelstein2015increasing} y de hacer una lista de las dudas surgidas y propuestas a dar ya he leido el artículo de \cite{wang2019dominant} sobre las galaxias a redshift $z>3$ descubiertas por ALMA las cuales parecen ser abundantes y masivas y no detectables por el HST. Esto tumbaría la susposición de \cite{finkelstein2015increasing} de la no existencia o poca abundancia de las galaxías sub-milimétricas lo que modificaría la función de luminosidad y por tanto la relación de masa halo - masa estelar ya que en el paper de \cite{finkelstein2015increasing} se basa principalmente en ella. La propuesta de lo que quiero hacer hoy es lo siguiente:
\begin{enumerate}
\item Terminar el análisis de \cite{finkelstein2015increasing}. $\surd$
\item Como encaja las nuevas observaciones de \cite{wang2019dominant} en el paper de \cite{finkelstein2015increasing}
\item Avanzar en el análisis de \cite{steinhardt2016impossibly}.
\item Plantear como incluir los planteamientos de \cite{wang2019dominant}:
\begin{enumerate}[i.]
\item Replantear la función de luminosidad de \cite{finkelstein2015increasing}.
\item Como encaja en el modelo jerárquico \citep{bower2006breaking}.
\item Las simulaciones de RAMSES nos pueden dar un orden distinto a la masa de los halos que encajen de una manera distinta de con la función de luminosidad. 
\end{enumerate}
\end{enumerate}