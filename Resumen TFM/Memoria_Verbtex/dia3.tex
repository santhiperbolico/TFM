\section*{13/03/2019}
Durante los días anteriores a esta fecha de reunión me propongo estudiar los papers de Rodríguez Puebla del 2017, las posibles soluciones al problema de \textit{Impossibly Early Galaxy Problem} (IEGP) planteado en \cite{steinhardt2016impossibly} y su posible estudio con las simulaciones de \cite{roca2016garrotxa}. Durante los días he leido un artículo \cite{montes2018intracluster} que estudia la posibilidad de usar la medición de la luz intercumular (ICL) de los cúmulos galácticos para el estudio de la masa del halo. Compara la estimación con la hecha por lentes gravitacionales (GL) y la medida por rayos X (RX) y se ve que a diferencia que la medida de RX, la cual discrepa en grandes medidas la del ICL y del GL en los escenarios de interaciónes gravitacionales entre galaxias y procesos de fusión, se ajusta muy bien a las medidas dadas por las GL. Este estudio solo llega a galaxias de z<1, la idea es ver si es viable usar este método para la determinación de la masa del halo en z>4 y si es así usarlo para la determinación de las masa de los halos estudiados por \citep[CANDELS]{grogin2011candels} y \citep[SPLASH]{capak2012splash}.\\

Volviendo al artículo de Rodríguez Puebla, tras analizar los abstracts de todos los papers en el que aparece como autor principal en los años 2017 y 2018 he encontrado \cite{rodriguez2017constraining} en el trata la relación masa estelar - masa halo. Esta discusión es de gran interés en el problema IEGP ya que uno de las posibles respuestas a la cuestión es la invalidez de los métodos que se usaron en la medición de la masa de los halos ya que trataban el ratio masa estelas- halo constante en el tiempo, lo que parece discutir dicho paper \citep{rodriguez2017constraining}. 
\begin{obs}
Una posible manera de plantear el TFG es abordar la problemática de la estimación del ratio masa estelar - masa halo, considerandola como una posible solución al problema del IEGP. Otra posible solución relacionada con el error observacional es el fallo en las estimaciones de los redshifts, lo que podría solucionar el problema de los AGN tempranos, pero eso aún no he podido leer nada más.
\end{obs}

